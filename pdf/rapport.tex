\documentclass[a4paper,twocolumn,french]{article}

% Set page dimentions
\usepackage[margin=2cm]{geometry}

% Packages for french documents
\usepackage{babel}
\usepackage[utf8]{inputenc}
\usepackage[T1]{fontenc}

% Define some colors
\usepackage{color}
\definecolor{string}{RGB}{100, 200, 0}
\definecolor{comment}{RGB}{150, 150, 150}
\definecolor{identifier}{RGB}{100, 100, 200}

% Source code style
\usepackage{listings}
\lstset{
	basicstyle=\footnotesize\ttfamily, % sets font style for the code
	frame=single,                 % adds a frame around the code
	showstringspaces=false,       % underline spaces within strings
	tabsize=4,                    % sets default tabsize to 2 spaces
	breaklines=true,              % sets automatic line breaking
	breakatwhitespace=true,       % sets if automatic breaks should only happen at whitespace
	keywordstyle=\color{magenta}, % sets color for keywords
	stringstyle=\color{string},   % sets color for strings
	commentstyle=\color{comment}, % sets color for comments
	emphstyle=\color{identifier}, % sets color for comments
}

% Hyperlinks
\usepackage[hyphens]{url}
\usepackage[hidelinks]{hyperref}

% Graphics
\usepackage{graphicx}
\graphicspath{ {img} }

\title{Projet ASTRE - Modélisation et Vérification de systèmes concurrents}
\date{Février 2021}
\author{Sylvain Joube \and Nicolas Peugnet}

\begin{document}

\maketitle

\tableofcontents

\section*{Introduction}

Bonjour !

\section{Première partie}

Cette fois j'ai un peu fait les choses bien,
déjà en mettant la documentclass en \emph{A4} et aussi en passant en mode \emph{twocolumn}
(toujours plus classe et lisible).
J'ai également un peu réduit les marges, parce que faut pas trop déconner non plus.

\begin{figure}[h]
	\centering
	\includegraphics[width=0.3\linewidth]{wtf.png} % changer ici l'image
	\caption{WTF ??\protect\footnotemark}
	\label{fig:wtf}
\end{figure}
\footnotetext{Cette image n'a été ajouté que pour avoir un template et pouvoir tester l'inclusion d'images}

\end{document}
